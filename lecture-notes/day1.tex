\documentclass{article}

\usepackage{listings}

\begin{document}
Note: during lectures have students install mobaXterm
\section{ROS In a Nutshell}
\begin{itemize}
    \item \textbf{Peer-to-Peer:} \quad Individual programs communicate over defined API
    \item \textbf{Distributed:} \quad Programs can run on any number of computers
    \item \textbf{Multilingual:} \quad Programs can be written in any language with a client library (Official: C++/Python, Unofficial: Java, Javascript, Rust, Julia)
    \item \textbf{Lightweight:} \quad Libraries are very thin/have low overhead
    \item \textbf{Free and Open Source:} \quad ROS in its entirety is open source
\end{itemize}
\section{ROS Master}
\begin{itemize}
    \item Manages comms between nodes (processes)
    \item Every node registers with the master at startup
    \item Started with \lstinline{roscore}.
\end{itemize}
\section{ROS Nodes}
\begin{itemize}
    \item Single purpose executable program
    \item Individually compiled, executed, managed
    \item Organized in \textit{packages}
    \item Run a node with \lstinline{rosrun <package_name> <node_name>}
    \item See active nodes with \lstinline{rosnode list}
    \item See information about a node with \lstinline{rosnode info <node_name>}
\end{itemize}
\section{ROS Topics}
\begin{itemize}
    \item Nodes communicate over \textit{topics}
    \begin{itemize}
        \item Nodes can \textit{publish} or \textit{subscribe} to a topic
        \item Typically, 1 publisher and $n$ subscribers
    \end{itemize}
    \item Topic is a name for a stream of messages
    \item List active topics with \lstinline{rostopic list}
    \item Subscribe and print the contents of a topic with \lstinline{rostopic echo <topic_path>}
    \item Show information about a topic with \lstinline{rostopic info <topic_path>}
\end{itemize}
\section{ROS Messages}
\begin{itemize}
    \item Data structure defining the \textit{type} of a topic
    \item Comprised of a nested structure of integers, floats, booleans, strings, etc. and arrays of objects
    \item Defined in \textit{*.msg} files
    \item See the type of a topic \lstinline{rostopic type <topic_path>}
    \item Publish a message to a topic \lstinline{rostopic pub <topic_path> <topic_type> <data_string>}
\end{itemize}
\section{ROS Workspace Environment}
\begin{itemize}
    \item Setup students with workspaces on odroid
    \item \lstinline{setup <name> <github_name> <github_email>}
    \item \lstinline{mkdir ws; cd ws; mkdir src; catkin_make; source devel/setup.bash}
    \item check it worked with \lstinline{echo $ROS_PACKAGE_PATH}
\end{itemize}
\section{Catkin Build System}
\begin{itemize}
    \item The src directory contains source code. This is where you clone, create, edit code.
    \item The build directory contains the code built whenever you run \lstinline{catkin_make}.
    \item The devel directory contains scripts to make ros scripts work with this workspace.
    \item Don't touch the build or devel directories. Just use the src directory.
\end{itemize}
\section{ROS Packages}
\begin{itemize}
    \item ROS software is organized into packages, which contains source code, launch files, config files, message definitions, data and docs
    \item Packages can depend on other packages
    \item To create a package, go to the src directory and run \lstinline{catkin_create_pkg <package_name> <dependencies>...}
    \item Packages have a CMakeLists.txt and a package.xml. We shouldn't need to touch the package.xml, unless we change what ROS dependencies we have.
    \item Go over CMakeLists.txt
\end{itemize}
\section{rospy example}
\begin{itemize}
    \item run \lstinline{catkin_create_pkg example1 rospy std_msgs}
    \item edit scripts/node
\end{itemize}
\begin{lstlisting}
#!/usr/bin/env python
import rospy
from std_msgs.msg import String

def run():
    pub = rospy.Publisher('test_wilson', String, 
                          queue_size=10)
    rospy.init_node('test_wilson')
    rate = rospy.Rate(10)
    while not rospy.is_shutdown():
        test_str = 'hello world {}'.format(
                        rospy.get_time())
        pub.publish(test_str)
        rate.sleep()

if __name__ == '__main__':
    try:
        run()
    except rospy.ROSInterruptException:
        pass
\end{lstlisting}
\begin{itemize}
    \item edit CMakeLists.txt
\end{itemize}
\begin{lstlisting}
catkin_install_python(PROGRAMS scripts/node 
DESTINATION ${CATKIN_PACKAGE_BIN_DESTINATION})
\end{lstlisting}
\section{rospy subscriber example}
\begin{itemize}
    \item edit scripts/listener
\end{itemize}
\begin{lstlisting}
#!/usr/bin/env python
import rospy
from std_msgs.msg import String

def callback(data: String):
    rospy.loginfo(rospy.get_caller_id() + 
                  "I heard {}".format(data.data))

def run():
    rospy.init_node('test_wilson_listen')
    rospy.Subscriber('test_wilson', String, callback)
    rospy.spin()

if __name__ == '__main__':
    try:
        run()
    except rospy.ROSInterruptException:
        pass
\end{lstlisting}
\begin{itemize}
    \item edit CMakeLists.txt
\end{itemize}
\begin{lstlisting}
catkin_install_python(PROGRAMS scripts/listener 
DESTINATION ${CATKIN_PACKAGE_BIN_DESTINATION})
\end{lstlisting}

\end{document}